% additional options: [seceqn,secthm,crcready]
\documentclass[sat]{iosart2x}

%% Packages
\usepackage{dcolumn}

%% Definitions
\newcolumntype{d}[1]{D{.}{.}{#1}}

%% Article Info
\firstpage{1}
\lastpage{6}
\volume{1}
\pubyear{2021}

\begin{document}
\begin{frontmatter} % The preamble begins here.

%
%\pretitle{Pretitle}
\title{Instructions for the preparation of a camera-ready paper in \LaTeX\thanks{Footnote in title.}}

\runtitle{Instructions for the preparation of a camera-ready paper in \LaTeX}
%\subtitle{Subtitle}

\begin{aug}
\author[A]{\inits{F.}\fnms{First} \snm{Author}\ead[label=e1]{editorial@iospress.nl}}
\author[B]{\inits{S.}\fnms{Second} \snm{Author}\ead[label=e2]{second@somewhere.com}}
\author[B]{\inits{T.}\fnms{Third} \snm{Author}\ead[label=e3]{third@somewhere.com}}
\address[A]{Journal Production Department, \orgname{IOS Press}, Nieuwe Hemweg 6b, 1013 BG, Amsterdam, \cny{The~Netherlands}}
\address[B]{Department first, \orgname{University or Company name},
Abbreviate US states, \cny{Country}}
\end{aug}

\begin{abstract}
The abstract should be clear, descriptive, self-explanatory and no longer than 200 words. It should also
be suitable for publication in abstracting services. Do not include references or formulae in the abstract.
\end{abstract}

\begin{keyword}
\kwd{Keyword one}
\kwd{keyword two}
\kwd{keyword three}
\kwd{keyword four}
\kwd{keyword five}
\end{keyword}
\end{frontmatter}




\section{Introduction}

The instructions are designed for the preparation of a camera-ready and accepted paper in \LaTeX{} and should be read carefully.
Prepare your paper in the same style as used in this sample pdf file.
These instructions also contain the necessary information for manual editing.

Manuscripts must be written in English. Authors whose native language is not English are recommended to seek the advice of a native English speaker,
if possible, before submitting their manuscripts. In the text no reference should be made
to page numbers; if necessary, one may refer to sections. Try to avoid excessive use of italics and bold face.


\section{Structure of the paper}
\subsection{Title page}

The title page should provide the following information:

\begin{itemize}
\item Title (should be clear, descriptive and not too long).
\item  Name(s) of author(s).
\item  Full affiliation(s). Names of main institution and country should be given within respective tags: \verb|\orgname{}|, \verb|\cny{}|.
\item  Abstract should be clear, descriptive, self-expla\-na\-tory and no longer than 200
words. It should also be suitable for publication in abstracting services.
\item  Up to five keywords.
\end{itemize}

\subsection{Body of the text}
\begin{itemize}
\item The use of first persons (i.e., ``I'', ``we'', ``their'', possessives, etc.) should be avoided,
and can preferably be expressed by the passive voice or other ways. This also applies to the Abstract.
\item A research paper should be structured in terms of four parts, each of which may comprise of multiple sections:
\begin{itemize}
\item Part One is problem description/definition, and a literature review upon the state of the
art.
\item Part Two is methodological formulation and/or theoretical development (fundamentals, principle and/or approach,
etc.).
\item Part Three is prototyping, case study or experiment.
\item Part Four is critical evaluation against related works, and the
conclusion.
\end{itemize}
\end{itemize}

In any article it is unnecessary to have an arrangement statement at the beginning (or end) of every \mbox{(sub-)} section.
Rather, a single overall arrangement statement about the whole paper can be made at the end of the Introduction section.


\section{Typographical style and layout}

\subsection{Type area}

The \texttt{iosart2x.cls} document class has been designed to produce
the right layout from your \LaTeX{} input. Authors are requested to
strictly follow these instructions. \emph{The provided class file
iosart2x must not be changed}.

The text output area is automatically set within an area of particular journal layout measurements.
Please do not use any
\LaTeX{} or \TeX{} commands that affect the layout or formatting of
your document (i.e. commands like \verb|\textheight|,
\verb|\textwidth|, etc.).

\subsection{Font}

For literal text, please use
\texttt{type\-writer} (\verb|\texttt{}|)
or \textsf{sans serif} (\verb|\textsf{}|). \emph{Italic} (\verb|\emph{}|)
or \textbf{boldface} (\verb|\textbf{}|) should be used for emphasis.


\subsection{General layout}

For the main
body of the paper use the commands of the standard \LaTeX{}
``article'' class. You can add packages or declare new \LaTeX{}
functions if and only if there is no conflict between your packages
and the \texttt{iosart2x.cls}.

Always give a \verb|\label| where possible and use \verb|\ref| for cross-referencing.

Use class option ``\texttt{crcready}'' in order to remove the page numbers from your article.


\subsection{Title page}

Use sentence case for the title.

Use \verb|\thanks{}| command for footnotes in \verb|\title|
and \verb|\author| commands.

Do not use capitals for author's surname.
Do not add a period after the last keyword.


\subsection{(Sub-)section headings}

Use the standard \LaTeX{} commands for headings: \verb|\section|, \verb|\subsection|, \verb|\subsubsection|, \verb|\paragraph|.
Headings will be automatically numbered.

\subsection{Footnotes}
Footnotes should only be used if absolutely essential.
In most cases it is possible to incorporate the information in the text.
If used, they should be kept as short as possible.
The footnotes are numbered automatically. Footnotes within the text should be coded with the command
\verb|\footnote{Text}|.


\subsection{References}

References should be collected at the end of your paper (environment
\verb|thebibliography|). References should be listed alphabetically in the style presented in the section \textbf{References} at the
end of these instructions. Use the command \verb|\cite| to refer to the entries in the bibliography so that your
accumulated list corresponds to the citations made in the text body.

\begin{figure}[b]
\includegraphics{iospress}
\caption{Figure caption.}
\end{figure}


\subsection{Figures}
\subsubsection{General remarks on figures}
The text should include references to all figures.
Refer to figures in the text as Fig. 1 (or Figure 1 in the beginning of a sentence), Fig. 2, etc., \textbf{not} with the section number included,
e.g. Figure 2.3, etc. Do not use the words ``below'' or ``above'' when referring to the
figures.

Do not collect figures at the back of your article, but incorporate them in the text.

Position figures at the top or bottom of a page, near the paragraph in which the figure is first mentioned.
Figures should not have text wrapped alongside.

Each figure should have a self-explanatory caption. Place the figure caption \textit{below} the figure.

All figures coded with \verb|figure| and \verb|\caption| will be numbered automatically.

On maps and other figures where a scale is needed, use bar scales rather than numerical ones of the type 1:10,000.

\subsubsection{Quality of illustrations}\label{s3.8.2}
Do \textit{not} use illustrations taken from the Internet.
The resolution of images intended for viewing on a screen is not sufficient for the printed version of the journal.
If you are incorporating screen captures, keep in mind that the text may not be legible after reproduction
(using a screen capture tool, instead of the Print Screen option of PC's, might help to improve the quality).


\begin{itemize}
\item Line art should have a minimum resolution of 600 dpi;
\item  grayscales (incl photos) should have a minimum resolution of 300 dpi (no lettering), or 500 dpi (when there is lettering);
\item  do not save figures as JPEG, this format may lose information in the
process;
\item  do not use figures taken from the Internet, the resolution will be too low for
printing;
\item  do not use color in your figures if they are to be printed in black \& white, as this will reduce the print
quality (note that in image processing software the default is often color, so you should change the
settings);
\item  for figures that should be printed in color, please send a CMYK encoded files.
\end{itemize}

\subsubsection{Color figures}
It is possible to have figures printed in color,
provided the cost of their reproduction is paid for by the author.
Please contact \email{editorial@iospress.nl} for a quotation if you wish to have figures printed in color.

\subsection{Tables}

The text should include references to all tables.
Refer to tables in the text as Table 1, Table 2, etc., not with the section number included,
 e.g. Table 2.3, etc. Do not use the words ``below'' or ``above'' referring to the
 tables.

Position tables at the top or bottom of a page, near the paragraph in which the table is first mentioned.
Tables should not have text wrapped alongside.


Code your tables using \LaTeX{} environments \verb|table| and
\verb|tabular|. Each table should have a brief and self-explanatory caption that should be put above the
table. Do not use the period at the end of the table caption. Double column journal: if the table does not fit into one column it may be placed across both columns
using \verb|\begin{table*}|
so that it
appears at the top of a page.




All tables coded with \verb|table| and \verb|\caption| will be numbered automatically.

Column headings should be brief, but sufficiently explanatory.
Standard abbreviations of units of measurement should be added between parentheses.
Vertical lines should not be used to separate columns. Leave some extra space between the columns instead.
Any explanations essential to the understanding of the table should be given in footnotes at the bottom of the table.
SI units should be used, i.e., the units based on the metre, kilogramme, second, etc.

Tables should be presented in the form shown in
Table~\ref{t1}.  Their layout should be consistent
throughout.

\begin{table*}[b]
\caption{Table caption} \label{t1}
\begin{tabular}{@{}ll d{1.3} d{1.3} d{1.3} d{1.3} d{1.3} d{1.3}@{}}
\hline
Dataset&Models&\multicolumn{1}{c}{$\alpha_1$}&
\multicolumn{1}{c}{$\alpha_2$}&
\multicolumn{1}{c}{$\alpha_3$}&
\multicolumn{1}{c}{$\alpha_4$}&
\multicolumn{1}{c}{$\alpha_5$}&
\multicolumn{1}{c}{$\alpha_6$}\\
\hline
CSDS&Linear     &0.164&0.22&0.123&0.3&0.200&0.258\\
& Logistic      &0.189&0.155&0.157&0.201&0.154&0.144\\[6pt]
KCDS&Linear     &0.155&0.183&0.160&0.218&0.176&0.156\\
& Logistic      &0.187&0.125&0.151&0.184&0.187&0.125\\
\hline
\end{tabular}
\end{table*}



\subsection{Equations}

Do not put in equation numbers, since this is taken care of
automatically. The equation numbers are always consecutive and are
printed in parentheses flush with the right-hand margin of the text and
level with the last line of the equation.
Refer to equations in the text as Eq. (1), Eqs (3) and (5).


\section{Producing of the PDF document}

The \LaTeX\ document should be compiled using a Lua\LaTeX\ compiler.


\section{Fine tuning}
\subsection{Type area}
Check once more that all the text and illustrations are inside the type area and
that the type area is used to the maximum.

\subsection{Capitalization}
Use sentence case in the title and the headings.

\subsection{Running headlines}
Use \verb|\runtitle{}| command to insert short title that will appear at the top of each page. Author part will be generated automatically.

\section{Submitting the paper}
Submit the following to the journal Editorial office, online submission form or Editor-in-Chief (whichever is applicable for the journal):

\begin{enumerate}
\item The main \LaTeX{} document as well as other required files.
\item PDF version of the \LaTeX{} file.
\item Please make sure you do not submit more than one version of any item.
\end{enumerate}


\begin{acks}
Please include your acknowledgements in \verb|acks| environment. For short acknowledgment you may use \verb|ack| environment.
\end{acks}

\begin{appendix}
\section{Appendix title}
Appendices should be provided in \verb|appendix| environment, before References.

\subsection{First subsection}
Use the standard \LaTeX{} commands for headings in Appendix.

\subsection{Second subsection}
Headings will be automatically numbered.
\end{appendix}


\begin{thebibliography}{0}

\bibitem{1} L. Lamport, \textit{\LaTeX{} User's Guide \& Reference Manual},
 Addison Wesley Publishing Co, 1985.

\bibitem{2} B. Newman and E.T. Liu, Perspective on BRCA1, \textit{Breast Disease} \textbf{10} (1998), 3--10. \bid{doi={10.3233/BD-1998-101-203}}

\bibitem{3} D.F. Pilkey, Happy conservation laws, in: \textit{Neural Stresses}, J. Frost, ed., Controlled Press, Georgia, 1995, pp. 332--391.

\bibitem{4} E. Wilson, Active vibration analysis of thin-walled beams, Ph.D. Dissertation, University of Virginia, 1991.

\end{thebibliography}


\end{document}
