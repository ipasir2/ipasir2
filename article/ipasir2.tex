\documentclass[sat]{iosart2x}

\pubyear{2025}
\volume{1}
\firstpage{1}
\lastpage{1}

\begin{document}

\begin{frontmatter}
\title{IPASIR-2: Re-entrant Incremental SAT Solver API 2.0}
\runtitle{IPASIR-2}

\begin{aug}
\author[A]{\fnms{Markus} \snm{Iser}\ead[label=e1]{markus.iser@kit.edu}}
\author{\fnms{Felix} \snm{Kutzner}\ead[label=e2]{felix.kutzner}}
\author[A]{\fnms{Dominik} \snm{Schreiber}\ead[label=e3]{dominik.schreiber@kit.edu}}
\author[B]{\fnms{Katalin} \snm{Fatzekas}\ead[label=e3]{katalin.fatzekas}}
\author[C]{\fnms{Armin} \snm{Biere}\ead[label=e3]{armin.biere}}
\address[A]{Institute of Theoretical Informatics, \orgname{KIT}, \cny{Germany}}
\address[B]{Department first, \orgname{University or Company name}, Abbreviate US states, \cny{Country}}
\address[C]{Department first, \orgname{University or Company name}, Abbreviate US states, \cny{Country}}
\end{aug}

\begin{abstract}
This paper describes the second version of the incremental SAT solver interface IPASIR, which is a major revision of the first version.
IPASIR-2 introduces error codes, a solver configuration interface, and methods to support clause sharing and proof logging.
\end{abstract}

\begin{keyword}
\kwd{incremental}
\kwd{satisfiability}
\kwd{solver}
\kwd{interface}
\end{keyword}
\end{frontmatter}


\section{Introduction}

IPASIR~\cite{ipasir} is a C API for incremental SAT solvers.
This paper describes the second version of the IPASIR interface, which is a major revision of the first version.
The following major requirements have been addressed in IPASIR-2.

\paragraph{Error Codes}
The introduction of error codes makes it possible to distinguish between different types of error.
All functions in IPASIR-2 return an error code and data is returned via output parameters.

\paragraph{Solver Configuration Interface}
Many SAT solvers have a large number of configuration options.
IPASIR-2 introduces a solver configuration interface to allow these options to be set.
Some configuration options are universal and therefore specified in the standard, others are solver specific.
The interface is designed to be generic and extensible to support a wide range of solver configuration options.

\paragraph{Clause Sharing and Proof Logging}
While IPASIR already supported clause export callbacks, IPASIR-2 introduces clause import and clause delete callbacks.
This allows IPASIR-2 solvers to be used in clause sharing portfolios and to log drat proofs.
For more advanced proof logging, e.g. LRAT, IPASIR-2 allows proof metadata to be passed with each clause.
This metadata is method specific and can be configured by the user.

%%%%%%%%%%%%%%%%%%%%%%%%%%%%%%%%%%%
\section{Interface Description}
%%%%%%%%%%%%%%%%%%%%%%%%%%%%%%%%%%%

This section describes the IPASIR-2 user interface.
We divide the interface into the following functional categories: core functions, solver configuration, clause sharing and proof logging, and fact sharing.
Each category is described in a separate subsection.
In addition to a functional specification, we provide a brief rationale for the design choices made.

%%%%%%%%%%%%%%%%%%%%%%%%%%%%%%%%%%%
\subsection{General Definitions}

\subsubsection{Thread Safety}

The IPASIR-2 specification allows multiple solvers to be initialised and used in parallel.
The intention is to cover use cases where the client starts an arbitrary number of solver instances in parallel, but interacts with each solver instance only sequentially.
All IPASIR-2 functions that take a solver instance $S$ as argument must not be called on the same solver instance $S$ concurrently.
If $S$ calls a callback function, the client may only access $S$ from the callback of the calling thread.
Solver instances are only allowed to call callback functions during \texttt{ipasir2\_solve}, and only from the same thread as the client calling \texttt{ipasir2\_solve}.
This is to keep the client code simple and to avoid compatibility issues, for example when IPASIR-2 is used in scripting languages.
IPASIR-2 implementations may provide custom options to replace these thread safety requirements.

\subsubsection{Error Codes}

IPASIR-2 functions return error codes to indicate the success or failure of a function call.
The error codes are defined in the \texttt{ipasir2\_errorcode} enum type.
The following error codes are defined:

\begin{itemize}
    \item \texttt{IPASIR2\_E\_OK}: Success.\\
    The function call was successful.
    \item \texttt{IPASIR2\_E\_UNKNOWN}: Unknown error.\\
    The function call failed for an unknown reason.
    \item \texttt{IPASIR2\_E\_UNSUPPORTED}: Unsupported function.\\
    The function is not implemented by the solver.
    \item \texttt{IPASIR2\_E\_UNSUPPORTED\_ARGUMENT}: Unsupported argument.\\
    The function is not implemented for handling the given argument value.
    \item \texttt{IPASIR2\_E\_UNSUPPORTED\_OPTION}: Unknown option.\\
    The configuration option is not implemented by the solver.
    \item \texttt{IPASIR2\_E\_INVALID\_STATE}: Invalid state.\\
    The function call is not allowed in the current state of the solver.
    \item \texttt{IPASIR2\_E\_INVALID\_ARGUMENT}: Invalid argument.\\
    The given argument value is invalid.
    \item \texttt{IPASIR2\_E\_INVALID\_OPTION\_VALUE}: Invalid option value.\\
    The option value is outside the allowed range.
\end{itemize}

\subsubsection{Solver States}

The state of the IPASIR-2 solver is defined by the state of the underlying state machine.
State transitions are triggered by IPASIR-2 function calls.
The state machine is initialized in the CONFIG state.
This is new in IPASIR-2 and allows for setting configuration options before adding clauses.
Functions are only allowed to be called in the states specified in the documentation.
If a function is called in the wrong state, the function returns \texttt{IPASIR2\_E\_INVALID\_STATE}.
The following states are defined:

\begin{itemize}
    \item \texttt{IPASIR2\_S\_CONFIG}: Configuration state.
    \item \texttt{IPASIR2\_S\_INPUT}: Input state.
    \item \texttt{IPASIR2\_S\_SAT}: Satisfiable state.
    \item \texttt{IPASIR2\_S\_UNSAT}: Unsatisfiable state.
    \item \texttt{IPASIR2\_S\_SOLVING}: Solving state.
\end{itemize}

Another innovation in IPASIR-2 is that states are partially ordered.
This allows for specifying a maximal state in which a function can be called, or in which a configuration option can be set.
The partial order is defined as $\mathtt{CONFIG < INPUT = SAT = UNSAT < SOLVING}$.

%%%%%%%%%%%%%%%%%%%%%%%%%%%%%%%%%%%
\subsection{Core Functions}

\begin{figure}[h]
    \tt\raggedright
    ipasir2\_signature(char** signature)\\[1ex]
    ipasir2\_init(void** S)\\
    ipasir2\_release(void* S)\\[1ex]
    ipasir2\_add(void* S, int32\_t* clause, int len, int forgetable, void* proofmeta)\\
    ipasir2\_solve(void* S, int32\_t* assumps, int len, int* result)\\
    ipasir2\_value(void* S, int32\_t lit, int* result)\\
    ipasir2\_failed(void* S, int32\_t lit, int* result)\\[1ex]
    ipasir2\_set\_terminate(void* S, void* data, int (*callback)(void* data))
    \caption{IPASIR-2 Core Functions}
\end{figure}

The core functions of IPASIR-2 are essentially the same as those of IPASIR.
A major difference is that all functions return an error code which is \texttt{IPASIR2\_E\_OK} if the fThe function \texttt{ipasir2\_init} constructs a new solver instance and returns a pointer to it in the output parameter.
The state of the returned solver is initialized to \texttt{CONFIG}.
The function \texttt{ipasir2\_signature} returns the name and version of the IPASIR-2 solver in the output parameter.
It can be called at any time and also simultaneously from any thread.

The \texttt{ipasir2\_init} function constructs a new solver instance and returns a pointer to it in the output parameter.
The state of the returned solver is initialised to \texttt{CONFIG}.
Multiple solver instances can be created in parallel.
The \texttt{ipasir2\_release} function destroys the given solver instance, freeing all solver resources and allocated memory.

The \texttt{ipasir2\_add} function adds a \texttt{clause} to the formula.
The clause is a pointer to an array of literals of the given \texttt{length}.
If \texttt{forgettable} is set to zero, the solver guarantees that the clause will be satisfied in any model it finds, otherwise the solver is allowed to remove the clause from the formula.
The \texttt{proofmeta} parameter points to a struct containing additional proof metadata.
The struct type and its semantics are specific to the selected proof method and are specified in the configuration options.

The function \texttt{ipasir2\_solve} solves the formula under the given \texttt{assumption} literals.
As long as the function is running, the solver is in the \texttt{SOLVING} state.
If the solver calls one of the callback functions during the execution of \texttt{ipasir2\_solve}, the state of the solver is also \texttt{SOLVING}.
Callbacks are allowed to execute solver functions that are allowed in the \texttt{SOLVING} state.
The function returns the result of the search in the output parameter \texttt{result}.
If the formula is satisfiable, the output parameter result is set to $10$ and the solver state is changed to \texttt{SAT}.
If the formula is unsatisfiable, the output parameter result is set to $20$ and the solver state is changed to \texttt{UNSAT}.
If the search is aborted, the output parameter result is set to $0$ and the solver state is changed to \texttt{INPUT}.

The \texttt{ipasir2\_value} function can only be used when the solver is in the \texttt{SAT} state and returns the truth value of the given literal in the satisfying assignment found.
The \texttt{result} output parameter is set to \texttt{lit} if \texttt{lit} is satisfied by the model, and to \texttt{-lit} if \texttt{lit} is not satisfied by the model.
The result output parameter can be set to zero if the assignment found supports both values for the literal.

The function \texttt{ipasir2\_failed} can only be used when the solver is in the \texttt{UNSAT} state and indicates whether the given assumption \texttt{literal} was used to prove unsatisfiability.
The output parameter \texttt{result} is set to $1$ if the given assumption literal was used to prove unsatisfiability, otherwise it is set to $0$.
The set of assumption literals for which the result of this function call is $1$ forms an unsatisfiable core of the formula.

The function \texttt{ipasir2\_set\_terminate} sets a callback function that is used to indicate a termination requirement to the solver.
The solver calls this function periodically while in the \texttt{SOLVING} state.
When the callback function is called, the given opaque \texttt{data} pointer is passed to the callback as its first argument.
If the callback function returns a non-zero value, the solver terminates the search.

%%%%%%%%%%%%%%%%%%%%%%%%%%%%%%%%%%%
\subsection{Configuration Interface}

\subsubsection{The Option Struct}

The \texttt{ipasir2\_option} struct is used to define solver configuration options.

\paragraph{NAME}
In IPASIR-2, options are identified by a string using dot-separated namespaces for structuring.
The top level namespace ``ipasir'' is reserved for the options specified in the IPASIR-2 standard.
If an IPASIR-2 solver supports a standard option, it is guaranteed to have the name specified in the standard.
It is recommended, but not mandatory, that IPASIR-2 solvers support all standard options.
Other proprietary options may be specified in the solver documentation, but must not be in the ``ipasir'' namespace.

\paragraph{MIN and MAX values}
The `min' and `max' fields specify the range of values that the option can take.
To simplify the interface, all option values are given as 64-bit integers.
Solvers using different option types must map them internally.

\paragraph{MAX\_STATE}
The `max\_state' field specifies the maximum solver state in which the option can be set.
While some options can be set at any time, e.g., from a callback during the \texttt{SOLVING} state, others can only be set in the \texttt{CONFIG} or \texttt{INPUT} states.

\paragraph{TUNABLE}
The `tunable' field indicates whether the option is eligible for tuning by an automated configuration tuning algorithm.

\paragraph{INDEXED}
Many options are global and can only be set once for the solver.
However, some options can be set per variable or other types of indices.
The `indexed' field indicates whether the option can be set per variable or other types of indices.
In this case the index parameter of \texttt{ipasir2\_set\_option} specifies the index.

\paragraph{HANDLE}
The `handle' is used by the option setter to identify the option.

\subsubsection{Option Getters and Setters}

\begin{figure}[h]
    \tt\raggedright
    ipasir2\_options(void* S, ipasir2\_option** options, int* count)\\
    ipasir2\_set\_option(void* S, ipasir2\_option* handle, int64\_t value, int64\_t index)
    \caption{IPASIR-2 Configuration Interface}
\end{figure}

The \texttt{ipasir2\_options} function returns an array of configuration options supported by the solver in the \texttt{options} output parameter, and the \texttt{count} output parameter contains the number of elements in the array.
The array is owned by the solver and must not be freed by the caller.

The \texttt{ipasir2\_set\_option} function sets the \texttt{value} of the option specified by the given \texttt{handle}.
If the option is indexed, the \texttt{index} parameter specifies the variable on which the option is set, otherwise the index parameter is ignored.
The value must be in the allowed range as specified in the option handle.
The solver must be in a state where the option can be set as specified in the option handle.

%%%%%%%%%%%%%%%%%%%%%%%%%%%%%%%%%%%
\subsection{Clause Sharing Interface}

\begin{figure}[h]
\tt\raggedright
ipasir2\_set\_export(void* S, void* data, int max\_length, void (*callback)(void* data, int32\_t const* clause, int32\_t len, void* proofmeta))\\
ipasir2\_set\_import(void* S, void* data, void (*callback)(void* data))\\
ipasir2\_set\_delete(void* S, void* data, void (*callback)(void* data, int32\_t const* clause, int32\_t len, void* proofmeta))
\end{figure}

The function \texttt{ipasir2\_set\_export} sets a callback function to receive learned clauses from the solver.
The solver calls this callback function in the \texttt{SOLVING} state for each learned clause whose size is less than \texttt{max\_length} or if \texttt{max\_length} is $-1$.
The opaque \texttt{data} pointer is passed to the callback function as its first parameter.
The \texttt{clause} parameter is a pointer to the learned clause of length \texttt{len} and is only valid during the execution of the callback function.
The \texttt{proofmeta} parameter points to a struct containing additional proof metadata.
The struct type and its semantics are specific to the selected proof method and are specified in the configuration options.

The function \texttt{ipasir2\_set\_import} sets a callback function for importing a clause into the solver.
This callback is called periodically while the solver is in the \texttt{SOLVING} state.
The opaque \texttt{data} pointer is passed to the callback function as its first parameter.
The callback function uses \texttt{ipasir2\_add} to add a clause to the solver.
This roundtrip is necessary to resolve any ownership issues with the clause pointer.
To import more than one clause, the callback must be called several times.
If there are no more clauses to import, the callback returns without calling \texttt{ipasir2\_add}.

The function \texttt{ipasir2\_set\_delete} sets a callback function for notification of deleted clauses.
The solver calls this function in the \texttt{SOLVING} state for each clause deleted from the formula.
The opaque \texttt{data} pointer is passed to the callback function as its first parameter.
The \texttt{clause} parameter is a pointer to the deleted clause of length \texttt{len} and is only valid during the execution of the callback function.
The \texttt{proofmeta} parameter points to a struct containing additional proof metadata.
The struct type and its semantics are specific to the selected proof method and are specified in the configuration options.

%%%%%%%%%%%%%%%%%%%%%%%%%%%%%%%%%%%
\subsection{Fact Sharing Interface}

\begin{figure}[h]
\tt\raggedright
ipasir2\_set\_fixed(void* S, void* data, void (*callback)(void* data, int32\_t fixed))
\caption{IPASIR-2 Fact Sharing Interface}
\end{figure}

The function \texttt{ipasir2\_set\_fixed} sets a callback to notify about fixed assignments.
The solver calls this function while being in the \texttt{SOLVING} state, whenever it determines that a literal true in all models of the formula.
The function does not return the fixed literals in any particular order, nor does it guarantee that all fixed literals are reported.

\bibliographystyle{ios1}
\bibliography{ipasir2}

\end{document}
